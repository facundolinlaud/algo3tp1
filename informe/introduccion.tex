\section{Análisis del problema}
En este documento se analizará el problema de la suma de subconjuntos (\textit{subset sum} en inglés). Dado un conjunto de $n$ elementos $V$ generalizados $v_{i}$ y un valor objetivo $T$, decidir si existe un subconjunto de S cuyos elementos sumen exactamente $T$ y, de existir múltiples soluciones, dar el mínimo cardinal. En el futuro, nos referiremos a las instancias de este problema con una tupla de la forma: $(V, n, T)$.

\vskip 8pt

Es interesante analizar este problema y sus derivados por su importante papel, por ejemplo, en las Ciencias de la Computación. Uno de sus caso de uso es encontrar la mejor distribución de tareas a ejecutar en dos procesadores, minimizando tiempos de \textit{idling} no es mas que una instancia del \textit{subset sum} con $T = \frac{total}{2}$, siendo $total$ la suma de los tiempos totales de todas las tareas a ejecutar.