\section{Técnicas propuestas}
Se nos pidió implementar tres soluciones para el problema utilizando las siguientes técnicas algorítmicas por separado:
\begin{itemize}
	\item Brute-forcing
	\item Backtracking
	\item Programación dinámica
\end{itemize}

\subsection{Brute-forcing}
Esta técnica consiste en probar absolutamente todas las combinaciones posibles. Si bien esta técnica es sencilla de implementar y asegura encontrar una solución al problema si es esta existe, su complejidad generalmente es muy cara y existen alternativas más eficientes que veremos a continuación.

\subsection{Backtracking}

\subsection{Programación Dinámica}
Esta técnica puede ser resumida como \textit{divide, conquer \& memoization}, donde se minimizan las llamadas recursivas al extinguir aquellas que ya han sido calculadas anteriormente. Esto sucede cuando existen recursiones que se invocan más de una vez con los mismos parámetros.

\vskip 8pt

En este documento, se considerarán dos enfoques de la programación dinámica:
\begin{itemize}
	\item El enfoque top-down
	\item El enfoque bottom-up
\end{itemize}