\section{Programación dinámica}
\subsection{Solución}
Antes que nada, debemos demostrar que el Principio de Optimalidad de Bellman vale para el problema de la suma de subconjuntos. Es decir, que toda solución óptima de un problema es construida a partir de subsoluciones óptimas de sus respectivos subproblemas. ¿Entonces, se cumple el principio de optimalidad? Verifiquémoslo.

\subsubsection{Cumplimiento del Principio de Optimalidad}
Asumiendo $I$ un conjunto solución de $n$ índices correspondientes a elementos en $V$ tal que $I=\{1, ..., n\}$ y $\sum_{i \in I}v_{i} = T$ con $T$ el valor objetivo. Para cada valor posible en $V$ hay dos posibilidades: que sea parte de la solución o no lo sea.

\begin{enumerate}[I)]
\item $n \notin I \implies I \subseteq \{1, 2, ..., n-1\}$
	\vskip 0pt
	De esta manera, el valor de $T$ no cambia. Es decir: $\sum_{i \in I}v_{i}=T$ con $I$ solución óptima por hipótesis. Luego, $I$ remanece solución óptima para el subproblema de valor objetivo $T$.
\item $n \in I:$
	\vskip 0pt
	En este caso, se tiene que cumplir $\sum_{i \in I^\prime}v_{i}=T-v_{i}$ con $I^\prime=I-\{n\}$ y $I^\prime$ solución óptima.
	\vskip 8pt
	Si $I^\prime=I-\{n\}$ \textbf{no} fuese solución óptima entonces existiría un $I=I^{\prime\prime}$ mínimo (y óptimo) tal que $|I^{\prime\prime}| < |I^\prime|$ y $\sum_{i \in I^{\prime\prime}}v_{i} = T - v_{n}$
	\vskip 8pt
	Pero si $I^{\prime\prime}$ fuese solución óptima del subproblema $n-1$ entonces la solución óptima del problema $n$ sería $I = I^{\prime\prime} \cup \{n\}$ y recordando la equivalencia $I=I^\prime \cup \{n\}$ que fue obtenida anteriormente, podemos generar un sistema de ecuaciones y luego aplicar álgebra:
	\vskip 8pt
	$
		\begin{cases}
			I^\prime = I - \{n\} \\ 
			I = I^{\prime\prime} \cup \{n\}
		\end{cases}
		\iff
		I^\prime = (I^{\prime\prime} \cup \{n\}) - \{n\} \iff I^\prime = I^{\prime\prime}
	$
	\vskip 8pt
	La equivalencia entre $I^\prime$ e $I^{\prime\prime}$ obtenida como conclusión contradice la declaración $|I^{\prime\prime}| < |I^\prime|$, obteniéndose un absurdo. Luego, $I^\prime$ es subsolución óptima del subproblema.
\end{enumerate}
Finalmente, el problema satisface el principio de optimalidad.
\vskip 8pt
$\qedwhite$

\subsection{Diseño}
\begin{center}
	\begin{minipage}{.8\textwidth}
		\begin{codebox}
			\Procname{$\proc{Subset-Sum}(i, accumulator)$}
				\li \If $(i, accumulator) \in dic$ \textbf{then}				\RComment $O(1)$
				\li \Then
					\If $i = 0$ \textbf{then}									\RComment $O(1)$
					\li \Then
						\If $accumulator = 0$ \textbf{then}						\RComment $O(1)$
						\li \Then
							 $dic[i][accumulator] \leftarrow 0$ 				\RComment $O(1)$
							  \li
						\Else
							\li $dic[i][accumulator] \leftarrow \infty$ 		\RComment $O(1)$
							\li
						\End
					\Else
						\li \If $V[i] > T$ \textbf{then}						\RComment $O(1)$
						\li \Then
							 $dic[i][accumulator] \leftarrow \func{Subset-Sum}(i - 1, accumulator)$
							 \li
						\Else
							\li $dic[i][accumulator] \leftarrow \func{min}\Big(\func{Subset-Sum}(i - 1, accumulator),$
							\Indentmore \li $1 + \func{Subset-Sum}(i - 1, accumulator - V[i])\Big)$
						\End
					\End
				\End

				\zi
				\li \Return $dic[i][accumulator]$								\RComment $O(1)$
			\End
		\end{codebox}

		\footnotesize Algoritmo
		\label{fig:algoritmo}
	\end{minipage}
\end{center}
Este algorítmo tiene un enfoque top-down. Es invocado con los parámetros $(i, accumulator)$ donde $i$ representa el i-ésimo elemento dentro de un conjunto $V$ de valores posibles utilizados para intentar sumar el valor objetivo $T$. La línea 1 pregunta si la dupla $(i, accumulator)$ está \textit{cacheada}, es decir, si ya fue calculada anteriormente. Si la respuesta es no, se procede a calcularlo. El siguiente condicional representa una de las condiciones necesarias para el caso base: que nos hayamos quedado sin elementos para analizar en el conjunto $V$. Si esto sucede, el algoritmo encontró una respuesta para este camino recursivo, pero depende del valor de $accumulator$:

\begin{itemize}
	\item Si $accumulator$ es cero, entonces la solución del camino recursivo es cero. Es decir, necesito 0 elementos del conjunto V para sumar 0.
	\item En cambio, si el valor de $accumulator$ no es cero, el camino no tiene solución, pues el camino recursivo no dispone de elementos suficientes para sumar $accumulator$.
\end{itemize}

Cualquiera sea la respuesta, esta se almacena en una matriz global $dic$ en los índices $(i, accumulator)$ para ser reutilizada la próxima vez que la función $Subset-Sum$ sea invocada con los mismos parámetros.

\vskip 8pt

Si el valor de $i$ es distinto de cero, entonces estamos en presencia de un paso recursivo. Aquí existen dos casos posibles:
\begin{itemize}
	\item $V[i]$ es mayor al valor acumulado al que se quiere llegar; es decir, el elemento del conjunto $V$ que se está analizando supera el valor de $accumulated$ por lo tanto no puede ser parte de la solución. En este, caso el algoritmo continúa por el camino recursivo donde $V[i]$ no es solución.
	\item $V[i]$ no es mayor al valor acumulado y podría llegar a ser parte de la solución. En este caso, se consideran y exploran ambos caminos: donde $V[i]$ es incluído en la solución y donde no lo es. El algoritmo se quedará con el menor de los dos caminos. Es decir, el de menor cardinalidad. Aquí es importante aclarar la importancia de devolver $\infty$ en el caso base sin solución explicado anteriormente.
\end{itemize}
Nuevamente, cualquiera sea el resultado del camino recursivo, este se guardará en el diccionario con los índices correspondientes. Finalmente, el algoritmo retornará lo que sea que haya guardado recientemente o no en los índices $(i, accumulator)$ del diccionario.

\subsection{Implementación}
\subsection{Complejidad}
\subsection{Análisis de performance}