\section{Conclusiones}
Podemos concluir, finalmente, que si bien todas las cotas teóricas son conocidas, el desempeño de ellos depende explícitamente de las características de los datos de entrada. Por ejemplo, la implementación que utiliza la técnica de Programación Dinámica parecía ser la más eficiente de las tres hasta que comenzaron a aparecer valores objetivos muy grandes y las técnicas de Fuerza Bruza y Back Tracking comenzaron a ganar terreno frente a la primera en cuestión. A la hora de evaluar una implementación frente a la otra, es necesario conocer exactamente qué tipo de entradas se desea procesar y cuál es su formato. Es posible, además, considerar mejoras en los algoritmos propuestos. Por ejemplo, uno podría partir en dos el valor de $T$ (por ende la complejidad en el algoritmo de Programación Dinámica se reduciría a la mitad) y buscar la suma de una de sus partes dentro de la lista siempre y cuando en el resto de los elementos no tomados también haya una suma de esta mitad. Esta alternativa es una instancia del Problema de la Suma de Subconjuntos y es un tópico que merece ser estudiado en profundidad.